
\addcontentsline{toc}{section}{\textcolor{cyan}{\textbf{Introduction générale}}}
\begin{flushleft}
\section*{\textcolor{cyan}{Introduction générale}}
 Pareil à notre vie en général, nos maisons se trouvent assez
 dotées de technologie. Nos habitats et les habitats du futur
 répondent à une probable insatisfaction innée de l’homme qui
 croit augmenter sa dominance sur son environnement par la
 technologie. On voit donc que sa maison répond à lui et à ses
 besoins. Ainsi, la technologie sert à la fois ses besoins, ses
 habitudes et son envie de confort. Elle prend en compte des
 situations significatives dans sa vie quotidienne : quitter son
 domicile, se réveiller dans un habitat chauffé, créer une ambiance
 désirée, avoir le café prêt et les volets ouverts.\newline
 
 Piloter notre bien-être, contrôler nos appareils et nos accès
 de près ou à distance en quelques clics. Construire un intérieur
 rassurant pour nous et nos proches afin de transformer notre
 maison en habitat moderne, intelligent et sécurisé est devenu un
 besoin de plus en plus exigé.\newline
 
 Pour cela il fallait rassembler et intégrer l’ensemble des
 techniques de l’électronique, de l’informatique, d’automatisme,
 de physique du bâtiment et des télécommunications afin de
 centraliser le contrôle de nos différents systèmes et sous-
 systèmes (volets roulants, porte de garage, portail d’entrée,
 prises électriques, chauffage, etc.). C'est autour de ce
 rassemblement et cette intégration que notre étude s’articule.
\end{flushleft}
\bibliographystyle{plain}
\newpage
	