
\addcontentsline{toc}{section}{\textcolor{cyan}{Conclusion générale}}
\begin{flushleft}
	\section*{\textcolor{cyan}{Conclusion générale}}
	Avec le développement continu des technologies de communication, des ordinateurs, des logiciels et des systèmes intelligents, les maisons connectées sont désormais une réalité concrète plutôt qu'une simple utopie. Cette évolution a considérablement amélioré le confort des gens dans leur maison et a permis l'offre de plusieurs nouveaux services tels que la sécurité et la protection des personnes, la surveillance accrue et l'amélioration du confort. Cette problématique a fait l'objet de nombreux travaux de recherche et notre projet de fin d'études intitulé "Contrôle et suivi d'une maison intelligente via Internet" nous a permis de jauger notre capacité à travailler en groupe, de mettre en valeur nos connaissances préalables et d'en acquérir de nouvelles. De plus, étant un sujet très récent et en constante évolution, cela nous permettra de poursuivre notre apprentissage à long terme.\newline
	
	Notre mémoire illustre le fonctionnement d'un système domotique basé sur Arduino utilisant deux technologies différentes (Bluetooth et WiFi) afin de surveiller et de contrôler les appareils domestiques. Nous avons développé une application Android en utilisant la plateforme Mit App Inventor et un site web à l'aide de l'outil Node-red qui est connecté à un broker Mosquitto installé sur une Raspberry Pi 4. Malgré la complexité du sujet, nous avons pu atteindre les trois objectifs principaux de ce projet : la commande via Bluetooth et WiFi, le contrôle de l'état des capteurs en temps réel et l'interface utilisateur.\newline
	
	Ce projet nous a permis de découvrir un nouveau domaine passionnant et innovant appelé domotique, qui nous a apporté énormément de connaissances. Nous pouvons dire que la période de réalisation de ce projet était une période éducative qui nous a permis d'explorer de nombreux domaines tels que l'internet des objets et le développement d'applications Android. Cependant, nous avons également rencontré plusieurs difficultés en raison de la nouveauté et de la complexité du sujet, ainsi que du respect des délais de réalisation.
\end{flushleft}

\newpage
