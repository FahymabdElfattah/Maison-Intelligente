\section*{\textcolor{cyan}{Programme arduino avec la communication WiFi ESP8266}}

\begin{flushleft}
	\begin{lstlisting}[style=CStyle]
		#include <ESP8266WiFi.h>
		#include <PubSubClient.h>
		
		//Relays for switching appliances
		#define Relay1            D6
		#define Relay2            D2
		#define Relay3            D1
		#define Relay4            D5
		#define Relay5            D3
		#define Relay6            D4
		#define Relay7            D7
		#define MSG_BUFFER_SIZE  (50)
		
		#define sub1 "lampe1"
		#define sub2 "lampe2"
		#define sub3 "lampe3"
		#define sub4 "lampe4"
		
		#define sub5 "lampe5"
		#define sub6 "Raise-Curtain"
		#define sub7 "Come-Down-Curtain"
		#define sub8 "Cooling"
		
		
		
		const char* ssid = "HTL-WS";
		const char* password = "1234567890";
		const char* mqtt_server = "XXX.XXX.XXX.XXX";  // Local IP address of Raspberry Pi
		
		const char* username = "MQTT_USERNAME";
		const char* pass = "MQTT_PASSWORD";
		
		
		char str_hum_data[10];
		char str_temp_data[10];
		char str_gaz_data[10];
		char msg[MSG_BUFFER_SIZE];
		
		int value = 0;
		unsigned long lastMsg = 0;
		
		WiFiClient espClient;
		PubSubClient client(espClient);
		
		void setup() {
			pinMode(Relay1, OUTPUT);
			pinMode(Relay2, OUTPUT);
			pinMode(Relay3, OUTPUT);
			pinMode(Relay4, OUTPUT);
			
			Serial.begin(115200);
			WiFi_setup();
			client.setServer(mqtt_server, 1883);
			client.setCallback(callback);
		}
		
		void loop() {
			
			if (!client.connected()) {
				reconnect();
			}
			client.loop();
			
			unsigned long now = millis();
			if (now - lastMsg > 2000) {
				float hum_data = random(-50,50);
				Serial.println(hum_data);
				/* 4 is mininum width, 2 is precision; float value is copied onto str_sensor*/
				dtostrf(hum_data, 4, 2, str_hum_data);
				float temp_data = random(0,100); // or dht.readTemperature(true) for Fahrenheit
				dtostrf(temp_data, 4, 2, str_temp_data);
				lastMsg = now;
				Serial.print("Publish message: ");
				Serial.print("Temperature - "); Serial.println(str_temp_data);
				client.publish("temp", str_temp_data);
				Serial.print("Humidity - "); Serial.println(str_hum_data);
				client.publish("hum", str_hum_data);
				Serial.print("Gaze - "); Serial.println(str_gaz_data);
				client.publish("gaz", str_gaz_data);
			}
		}
		
		
		//__________________________________________WiFi Conection______________________________
		void WiFi_setup() {
			delay(10);
			Serial.print("Connecting to ");
			Serial.println(ssid);
			
			WiFi.mode(WIFI_STA);
			WiFi.begin(ssid, password);
			
			while (WiFi.status() != WL_CONNECTED) {
				delay(500);
				Serial.print(".");
			}
			randomSeed(micros());
			Serial.println("");
			Serial.println("WiFi connected");
			Serial.println("IP address: ");
			Serial.println(WiFi.localIP());
		}
		
		void callback(char* topic, byte* payload, unsigned int length) {
			Serial.print("Message arrived [");
			Serial.print(topic);
			Serial.print("] ");
			
			if (strstr(topic, sub1)){
				for (int i = 0; i < length; i++) {
					Serial.print((char)payload[i]);
				}
				Serial.println();
				// Switch on the LED if an 1 was received as first character
				if ((char)payload[0] == '1') {
					digitalWrite(Relay1, HIGH);   // Turn the LED on (Note that LOW is the voltage level
					// but actually the LED is on; this is because
					// it is active low on the ESP-01)
				} else {
					digitalWrite(Relay1, LOW);  // Turn the LED off by making the voltage HIGH
				}
			}
			
			else if ( strstr(topic, sub2)){
				for (int i = 0; i < length; i++) {
					Serial.print((char)payload[i]);
				}
				Serial.println();
				// Switch on the LED if an 1 was received as first character
				if ((char)payload[0] == '1') {
					digitalWrite(Relay2, HIGH);   // Turn the LED on (Note that LOW is the voltage level
					// but actually the LED is on; this is because
					// it is active low on the ESP-01)
				} else {
					digitalWrite(Relay2, LOW);  // Turn the LED off by making the voltage HIGH
				}
			}
			else if ( strstr(topic, sub3)){
				for (int i = 0; i < length; i++) {
					Serial.print((char)payload[i]);
				}
				Serial.println();
				// Switch on the LED if an 1 was received as first character
				if ((char)payload[0] == '1') {
					digitalWrite(Relay3, HIGH);   // Turn the LED on (Note that LOW is the voltage level
					// but actually the LED is on; this is because
					// it is active low on the ESP-01)
				} else {
					digitalWrite(Relay3, LOW);  // Turn the LED off by making the voltage HIGH
				}
			}
			else if ( strstr(topic, sub4)){
				for (int i = 0; i < length; i++) {
					Serial.print((char)payload[i]);
				}
				Serial.println();
				// Switch on the LED if an 1 was received as first character
				if ((char)payload[0] == '1') {
					digitalWrite(Relay4, HIGH);   // Turn the LED on (Note that LOW is the voltage level
					// but actually the LED is on; this is because
					// it is active low on the ESP-01)
				} else {
					digitalWrite(Relay4, LOW);  // Turn the LED off by making the voltage HIGH
				}
			}
			
			else{
				Serial.println("unsubscribed topic");
			}
		}
		
		void reconnect() {
			// Loop until we're reconnected
			while (!client.connected()) {
				Serial.print("Attempting MQTT connection...");
				// Create a random client ID
				String clientId = "ESP8266Client-";
				clientId += String(random(0xffff), HEX);
				// Attempt to connect
				if (client.connect(clientId.c_str(), username, pass) ) {
					Serial.println("connected");
					// Once connected, publish an announcement...
					// ... and resubscribe
					client.subscribe(sub1);
					client.subscribe(sub2);
					client.subscribe(sub3);
					client.subscribe(sub4);
				} else {
					Serial.print("failed, rc=");
					Serial.print(client.state());
					Serial.println(" try again in 5 seconds");
					// Wait 5 seconds before retrying
					delay(5000);
				}
			}
		}
		
		
	\end{lstlisting}
\end{flushleft}

\newpage